\exercice*

\begin{align*}
  (( a )) x (( b|facteur("so") )) &= (( c )) x (( d|facteur("so") )) \\
  (( a )) x (( -c|facteur("so") ))x &= (( d )) (( -b|facteur("so") )) \\
  (* if a == c *)
    0 &= (( d - b )) \\
  (* else *)
    (* if a - c == 1 *)
      % Rien
    (* elif a - c == -1 *)
      -x &= (( d - b )) \\
      -1 \times -x &= -1 \times (( d - b )) \\
    (* else *)
      (( a - c )) x &= (( d - b )) \\
      x &= \frac{(( d - b ))}{(( a - c ))} \\
    (* endif *)
    x &
         (* if 100*(d-b)/(a-c) == (100*(d-b)/(a-c))|int *)
             =
         (* else *)
             \simeq
         (* endif *)
         (( ((d-b)/(a-c)) | facteur("2") ))
  (* endif *)
\end{align*}

(* if a == c and b == d *)
  Puisque $0=0$ est toujours vrai, l'équation a une infinité de solutions : tous les nombres réels.
(* elif a == c and b != d *)
  Puisque $0=((d - b))$ est toujours faux, l'équation n'a pas de solutions.
(* else *)
  L'unique solution est
  $x
  (* if 100*(d-b)/(a-c) == (100*(d-b)/(a-c))|int *)
      =
  (* else *)
      \simeq
  (* endif *)
  (( ((d-b)/(a-c)) | facteur("2") ))$.
(* endif *)
